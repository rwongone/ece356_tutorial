\input{configuration}

\title{Tutorial 1}

\author{Richard Wong \\ \small \texttt{rk2wong@edu.uwaterloo.ca}}
\institute{Department of Electrical and Computer Engineering \\
  University of Waterloo}
\date{\today}


\begin{document}

\begin{frame}
  \titlepage

\end{frame}

\begin{frame}
\frametitle{The Entity-Relationship Model}

In short, this model lets us represent:\\
\begin{itemize}
\item tables in a database with \textbf{mathematical sets}, and
\item queries on the database with a language called the \textbf{relational algebra}.
\end{itemize}

\end{frame}


\begin{frame}
\frametitle{Nouns of the Relational Model (1/2)}

A \textbf{relation} $R$ is equivalent to a table.

An \textbf{attribute} $A$ is equivalent to a column of a table.

We describe the combination of relation and attributes with a \textbf{relation schema} of the form $R(A_1, A_2, ... , A_n)$.

$Student(student\_number, name, address)$ tells us:\\
\begin{itemize}
\item there is a table called \texttt{Student}, and
\item it has attributes \texttt{name} and \texttt{address}.
\end{itemize}

\texttt{name} is an attribute of the relation \texttt{Student}.

\end{frame}

\begin{frame}
\frametitle{Nouns of the Relational Model (2/2)}

The \textbf{contents} of a relation $R$ are denoted $r(R)$.

$r(R)$ is an (unordered) set of (ordered) \textbf{n-tuples}.

Each \textbf{n-tuple} of $r(R)$ is equivalent to a row in the relation/table $R$.

The elements of each n-tuple correspond with the attributes $A_1, A_2, ... , A_n$ of $R$.

\end{frame}

\begin{frame}
\frametitle{Nouns of the Entity-Relationship Model (1/X)}

An \textbf{entity} [table] is equivalent to a relation, which is equivalent to a table.\\
\begin{itemize}
\item \texttt{Student} can be an entity in an E-R model.
\end{itemize}

However, we also use the term \textbf{entity} to describe an object in the real world, indistinguishable from other objects.\\
\begin{itemize}
\item A student with student number 20000000 and name Matt is an entity of type \texttt{Student}.
\end{itemize}

In this way, an entity [object] is equivalent to a row of an entity [table].

\end{frame}


\begin{frame}
\frametitle{Nouns of the Entity-Relationship Model (2/X)}

An \textbf{entity set} is a set of entities (in the object sense), and is equivalent to the contents of an entity table.\\
\begin{itemize}
\item The set of all students studying at the University of Waterloo might comprise the entity set \texttt{Student}.
\end{itemize}

For instance, \texttt{Student} = \{(20000000, Matt), (20000001, Josh)\}

\end{frame}


\begin{frame}
\frametitle{Nouns of the Entity-Relationship Model (3/X)}

Suppose we have another entity called \texttt{Course}, with attributes \texttt{course\_id} and \texttt{title}.

\texttt{Student} = \{(20000000, Matt), (20000001, Josh)\}\\
\texttt{Course} = \{(CS101, Intro to CS), (CS999, Super Hard CS)\}

A \textbf{relationship} describes some connection between entities, in both the table sense and the object sense.\\
\begin{itemize}
\item In our E-R model, \texttt{Student}s take \texttt{Course}s.
\item An individual student like Matt can take a course like Intro to CS.
\end{itemize}
\end{frame}

\begin{frame}
\frametitle{Nouns of the Entity-Relationship Model (4/X)}
Relationships are formalized as tuples of entities.

If we have the following entities:\\
$student_1$ = (20000000, Matt)\\
$course_1$ = (CS101, Intro to CS)

Then we might represent the fact that Matt takes Intro to CS with the relationship ($student_1$, $course_1$).

\end{frame}


\begin{frame}
\frametitle{Nouns of the Entity-Relationship Model (4/X)}

A \textbf{relationship set} is a set of relationships.\\
\begin{itemize}
  \item In an E-R model, a relationship set represents all the ways that entity tables are related.
  \begin{itemize}
    \item \texttt{Student}s take \texttt{Courses}.
    \item \texttt{Courses} have \texttt{Instructors}.
  \end{itemize}

  \item On an object level, a relationship set represents all of the relationships between individual objects.

  \begin{itemize}
    \item Matt takes Intro to CS.
    \item Josh also takes Intro to CS.
    \item Intro to CS is taught by Prof. X.
  \end{itemize}
\end{itemize}

In an E-R model, we can use a table to represent/contain a relationship set.\\
Such a table will have columns to identify which entities are related.\\
\begin{itemize}
  \item The \texttt{StudentTakesCourse} table might have columns for \texttt{student\_id} and \exttt{course\_id}.
\end{itemize}

\end{frame}
\end{document}

